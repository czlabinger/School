\section{Verschluesselung}

Symetrisch mittels einem Key

Asymetrische als verstaerkung mittels public/private key

Checksums z.B.: Sha256, Sha512, md5, \ldots

Bei veraenderung komplett anderer Hash

Es kann trozedem sein dass 2 verschiedene daten gleichen Hash haben

Bei Sha256 groessere wahrscheinlichkeit als bei Sha512 weil weniger bits

\section{BUS}

SPI I2C BUS 

BUS controller um collisions zu vermeiden

Wenn collisions -> Beide teilnehmer warten eine zufaellige Zeit

\section{Konsistenz in Datenbanken}
\subsection{Konsistenz}
\subsubsection{Einheitlich}
\paragraph{Datentyp}
Der Datentyp gibt an wie die Daten zu interpretieren sind und wie gross dieser ist.
Fuer Konsistenz ist es wichtig dass die Interpretation gleich oder mapbar ist.

\subparagraph{SQL subsprachen}
Data Definition Language
Data Manipulation Language
Data Query Language

\subsubsection{Regeln (Constraints)}
z.B: NotNull, Unique
Helfen bei Konsistenz koesten aber Zeit (Alles ueber 0.5s lade Zeit ist lange)

CRUD = Create Read Update Delete

\subsubsection{Deterministisch}
Computer sind deterministisch gebaut/programmiert wesshalb es eigendlich nicht moeglich ist zufaellige Zahlen zu generieren.
Desshalb wird PRNG (Pseudo Randon Number Generator) genutzt.
Dieser verwendet Sensoren, deren Stand nicht vorhergesehen werden kann, wie z.B.: CPU temperatur, Fan speed, ...

\section{Transaktionskonzepte}
ACID = Atomacity Consistency Isolation Durability

\paragraph{Atomar}
Atomar bedeutet, dass eine Transaktion entweder ganz durchgefuehrt wird oder garnicht.

\paragraph{CAP}
CAP = Consistency Availability Partitiontolerant
Es koennen hoechstens zwei dieser Ziele gleichzeitig erfuellt sein. Bei ACID muessen alle fier Ziele erfuellt sein.

Damit Daten konsitent sind muessen sie am Anfang und am Ende konsitent sein aber mussen bzw. koennen in der Mitte nicht konsitent sein.

\subsection{Isolation Levels (psql)}

\begin{tabular}{||c c c c||} 
    \hline
    Isolation Level & Diry Read & Nonrepeatable Read & Phantom Read & Serialization Anomaly \\ [0.5ex] 
    \hline
    
    \hline
    Read uncommited (Deadlocks moeglich) & Allowed, but not in PG & Possible & Possible & Possible \\ 
    \hline
    Read commited & Not possible & Possible & Possible & Possible \\
    \hline
    Repeatable read & Not possible & Not possible & Allowed, but not in PG & Possible \\
    \hline
    Serializable & Not possible & Not possible & Not possible & Not Possible \\
    \hline
\end{tabular}

Es gibt einen Unterschied zwischen Lesen zum anzeigen und Lesen zum schreiben. Beim Lesen zum schreiben muessen die gelesenen Daten gesperrt werden.

\subsection{Durability}
Dauerhaftes Spechern der Daten.
Wird sichergestellt dadurch dass die veraenderten Daten welche im RAM sind gespechert werden, dann auf die DB uebernommen werden und dann ueberprueft wird ob diese korrekt gespeichert werden koennen.
Falls dies nicht der fall ist wird ein rollback durchgefuehrt.

\subsection{Geschwindigkeit}
ACID ist am sichersten aber ist auch am langsamsten.
BASE = Basically Available Soft-state Eventually consistant

BA: Atomates speichern der Anfang in einer Queue und deren sequenzielle ausfuehrung.
S: Nach einer gewissen Zeit in der queue werden die Anfragen getimeouted.
E: Wenn alle Anfragen bearbeitet wurden ist die Datenbank konsitent.

Jede Anfrage kann von BASE mit der Hilfe von ACID bearbeitet werden.

\section{Build prozess}

.h/.hpp files (Header files) beinhalten die Struktur des dazugehoerigen .c/.cpp files.
Der inhalt der header files wird vom pre processor in das C/C++ file kopiert.
Da C nicht objekt orientiert ist gibt es kein method overloading.

\subsection{Prozess}
Das .h/.hpp file wird vom pre processor in das .c/.cpp file kopiert. Der compiler verwandelt dieses dann in .o/.obj files.
Der Assembler verwaltet das startup.a/.s file woraus der Linker dann ausfuehrbaren code macht welche mittels des Flash tools geflashed werden.

\section{CAP-Theorie}
Server = Diener

RPC, RMI, gRPC sind remote prozedure call implementationen welche fuer mittelware verwendet werden koennen.
MVC = Model View Controller
MVVM wird fuer web based implementations verwendet. 
DAO = Data Acess Object welches ueberprueft ob bestimmte Daten korrekt sind.

\subsection{CAP}
Consistency, Availability wurden bereits gemacht
Partition tolerance bedeutet das sub-systeme eigenstaendig funktionieren und die Daten nach einem Ausfall wieder miteinander syncen.
Es sind immer nur 2 der 3 CAP Ziele gleichzeitig moeglich.

\section{Timer, Delay, FPU}
\subsection{Cpu Cycles}
Ein delay kann mittels zaehlen von CPU Cycles hergestellt werden.
CPU clock speed / f/2
\subsection{Timer}
Timers haben ein Limit wie klein der delay sein kann wodurch ein offset entstehen kann der dann haendisch ausgebessert werden muss.
\section{MCP2515}
CAN Controller
CAN = Controler Area Network