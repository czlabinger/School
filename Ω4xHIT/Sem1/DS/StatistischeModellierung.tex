% Options for packages loaded elsewhere
\PassOptionsToPackage{unicode}{hyperref}
\PassOptionsToPackage{hyphens}{url}
%
\documentclass[
]{article}
\usepackage{amsmath,amssymb}
\usepackage{iftex}
\ifPDFTeX
  \usepackage[T1]{fontenc}
  \usepackage[utf8]{inputenc}
  \usepackage{textcomp} % provide euro and other symbols
\else % if luatex or xetex
  \usepackage{unicode-math} % this also loads fontspec
  \defaultfontfeatures{Scale=MatchLowercase}
  \defaultfontfeatures[\rmfamily]{Ligatures=TeX,Scale=1}
\fi
\usepackage{lmodern}
\ifPDFTeX\else
  % xetex/luatex font selection
\fi
% Use upquote if available, for straight quotes in verbatim environments
\IfFileExists{upquote.sty}{\usepackage{upquote}}{}
\IfFileExists{microtype.sty}{% use microtype if available
  \usepackage[]{microtype}
  \UseMicrotypeSet[protrusion]{basicmath} % disable protrusion for tt fonts
}{}
\makeatletter
\@ifundefined{KOMAClassName}{% if non-KOMA class
  \IfFileExists{parskip.sty}{%
    \usepackage{parskip}
  }{% else
    \setlength{\parindent}{0pt}
    \setlength{\parskip}{6pt plus 2pt minus 1pt}}
}{% if KOMA class
  \KOMAoptions{parskip=half}}
\makeatother
\usepackage{xcolor}
\usepackage[margin=1in]{geometry}
\usepackage{color}
\usepackage{fancyvrb}
\newcommand{\VerbBar}{|}
\newcommand{\VERB}{\Verb[commandchars=\\\{\}]}
\DefineVerbatimEnvironment{Highlighting}{Verbatim}{commandchars=\\\{\}}
% Add ',fontsize=\small' for more characters per line
\usepackage{framed}
\definecolor{shadecolor}{RGB}{248,248,248}
\newenvironment{Shaded}{\begin{snugshade}}{\end{snugshade}}
\newcommand{\AlertTok}[1]{\textcolor[rgb]{0.94,0.16,0.16}{#1}}
\newcommand{\AnnotationTok}[1]{\textcolor[rgb]{0.56,0.35,0.01}{\textbf{\textit{#1}}}}
\newcommand{\AttributeTok}[1]{\textcolor[rgb]{0.13,0.29,0.53}{#1}}
\newcommand{\BaseNTok}[1]{\textcolor[rgb]{0.00,0.00,0.81}{#1}}
\newcommand{\BuiltInTok}[1]{#1}
\newcommand{\CharTok}[1]{\textcolor[rgb]{0.31,0.60,0.02}{#1}}
\newcommand{\CommentTok}[1]{\textcolor[rgb]{0.56,0.35,0.01}{\textit{#1}}}
\newcommand{\CommentVarTok}[1]{\textcolor[rgb]{0.56,0.35,0.01}{\textbf{\textit{#1}}}}
\newcommand{\ConstantTok}[1]{\textcolor[rgb]{0.56,0.35,0.01}{#1}}
\newcommand{\ControlFlowTok}[1]{\textcolor[rgb]{0.13,0.29,0.53}{\textbf{#1}}}
\newcommand{\DataTypeTok}[1]{\textcolor[rgb]{0.13,0.29,0.53}{#1}}
\newcommand{\DecValTok}[1]{\textcolor[rgb]{0.00,0.00,0.81}{#1}}
\newcommand{\DocumentationTok}[1]{\textcolor[rgb]{0.56,0.35,0.01}{\textbf{\textit{#1}}}}
\newcommand{\ErrorTok}[1]{\textcolor[rgb]{0.64,0.00,0.00}{\textbf{#1}}}
\newcommand{\ExtensionTok}[1]{#1}
\newcommand{\FloatTok}[1]{\textcolor[rgb]{0.00,0.00,0.81}{#1}}
\newcommand{\FunctionTok}[1]{\textcolor[rgb]{0.13,0.29,0.53}{\textbf{#1}}}
\newcommand{\ImportTok}[1]{#1}
\newcommand{\InformationTok}[1]{\textcolor[rgb]{0.56,0.35,0.01}{\textbf{\textit{#1}}}}
\newcommand{\KeywordTok}[1]{\textcolor[rgb]{0.13,0.29,0.53}{\textbf{#1}}}
\newcommand{\NormalTok}[1]{#1}
\newcommand{\OperatorTok}[1]{\textcolor[rgb]{0.81,0.36,0.00}{\textbf{#1}}}
\newcommand{\OtherTok}[1]{\textcolor[rgb]{0.56,0.35,0.01}{#1}}
\newcommand{\PreprocessorTok}[1]{\textcolor[rgb]{0.56,0.35,0.01}{\textit{#1}}}
\newcommand{\RegionMarkerTok}[1]{#1}
\newcommand{\SpecialCharTok}[1]{\textcolor[rgb]{0.81,0.36,0.00}{\textbf{#1}}}
\newcommand{\SpecialStringTok}[1]{\textcolor[rgb]{0.31,0.60,0.02}{#1}}
\newcommand{\StringTok}[1]{\textcolor[rgb]{0.31,0.60,0.02}{#1}}
\newcommand{\VariableTok}[1]{\textcolor[rgb]{0.00,0.00,0.00}{#1}}
\newcommand{\VerbatimStringTok}[1]{\textcolor[rgb]{0.31,0.60,0.02}{#1}}
\newcommand{\WarningTok}[1]{\textcolor[rgb]{0.56,0.35,0.01}{\textbf{\textit{#1}}}}
\usepackage{graphicx}
\makeatletter
\def\maxwidth{\ifdim\Gin@nat@width>\linewidth\linewidth\else\Gin@nat@width\fi}
\def\maxheight{\ifdim\Gin@nat@height>\textheight\textheight\else\Gin@nat@height\fi}
\makeatother
% Scale images if necessary, so that they will not overflow the page
% margins by default, and it is still possible to overwrite the defaults
% using explicit options in \includegraphics[width, height, ...]{}
\setkeys{Gin}{width=\maxwidth,height=\maxheight,keepaspectratio}
% Set default figure placement to htbp
\makeatletter
\def\fps@figure{htbp}
\makeatother
\setlength{\emergencystretch}{3em} % prevent overfull lines
\providecommand{\tightlist}{%
  \setlength{\itemsep}{0pt}\setlength{\parskip}{0pt}}
\setcounter{secnumdepth}{-\maxdimen} % remove section numbering
\ifLuaTeX
  \usepackage{selnolig}  % disable illegal ligatures
\fi
\IfFileExists{bookmark.sty}{\usepackage{bookmark}}{\usepackage{hyperref}}
\IfFileExists{xurl.sty}{\usepackage{xurl}}{} % add URL line breaks if available
\urlstyle{same}
\hypersetup{
  pdftitle={StatistischeModellierung},
  pdfauthor={Zlabinger Christof},
  hidelinks,
  pdfcreator={LaTeX via pandoc}}

\title{StatistischeModellierung}
\author{Zlabinger Christof}
\date{2024-02-27}

\begin{document}
\maketitle

\subsection{Aufgabe 1}\label{aufgabe-1}

\subsubsection{Lade den Datensatz `state.x77' in R. Beschreibe die Daten
anhand der internen
Hilfe.}\label{lade-den-datensatz-state.x77-in-r.-beschreibe-die-daten-anhand-der-internen-hilfe.}

\begin{verbatim}
##    Population        Income       Illiteracy       Life Exp    
##  Min.   :  365   Min.   :3098   Min.   :0.500   Min.   :67.96  
##  1st Qu.: 1080   1st Qu.:3993   1st Qu.:0.625   1st Qu.:70.12  
##  Median : 2838   Median :4519   Median :0.950   Median :70.67  
##  Mean   : 4246   Mean   :4436   Mean   :1.170   Mean   :70.88  
##  3rd Qu.: 4968   3rd Qu.:4814   3rd Qu.:1.575   3rd Qu.:71.89  
##  Max.   :21198   Max.   :6315   Max.   :2.800   Max.   :73.60  
##      Murder          HS Grad          Frost             Area       
##  Min.   : 1.400   Min.   :37.80   Min.   :  0.00   Min.   :  1049  
##  1st Qu.: 4.350   1st Qu.:48.05   1st Qu.: 66.25   1st Qu.: 36985  
##  Median : 6.850   Median :53.25   Median :114.50   Median : 54277  
##  Mean   : 7.378   Mean   :53.11   Mean   :104.46   Mean   : 70736  
##  3rd Qu.:10.675   3rd Qu.:59.15   3rd Qu.:139.75   3rd Qu.: 81162  
##  Max.   :15.100   Max.   :67.30   Max.   :188.00   Max.   :566432
\end{verbatim}

Es handelt sich um eine Matrix mit 50 Zeilen und 8 Reihen welche Dated
der US Staaten beinhalten. Jede Zeile entspricht einem Staat. Diese
Reihen beinhalten die:

\begin{itemize}
\tightlist
\item
  Population im Jahr 1975 in 100 Einwohnern
\item
  Das Einkommen pro Person in 1974
\item
  Die Prozent an Analphabeten in 1970
\item
  Die Lebenserwartung von 1969-1971
\item
  Die Mordrate pro 100,000 Einwohnern
\item
  Die Prozent der Highschool Absolventen
\item
  Die Mittlere Anzahl an Tagen an denen es in der Hauptstadt oder in
  einer grossen Stadt, in den Jahren 1931-1960, es unter 0°c hatte.
\item
  Die Flaeche der Laender
\end{itemize}

\subsubsection{Ermittle ein lineares Regressionsmodell, dass die
Mordrate (`Murder') durch die unabhängigen Variablen Population, Income,
Illiteracy,und Life Exp(ectancy) erklärt. Schreibe die Modellgleichung
an und interpretiere die Werte der Koeffizienten im
Kontext.}\label{ermittle-ein-lineares-regressionsmodell-dass-die-mordrate-murder-durch-die-unabhuxe4ngigen-variablen-population-income-illiteracyund-life-expectancy-erkluxe4rt.-schreibe-die-modellgleichung-an-und-interpretiere-die-werte-der-koeffizienten-im-kontext.}

\begin{Shaded}
\begin{Highlighting}[]
\NormalTok{model }\OtherTok{\textless{}{-}}  \FunctionTok{lm}\NormalTok{(state.x77[,}\StringTok{"Murder"}\NormalTok{] }\SpecialCharTok{\textasciitilde{}}\NormalTok{ state.x77[,}\StringTok{"Population"}\NormalTok{] }\SpecialCharTok{+}\NormalTok{ state.x77[,}\StringTok{"Income"}\NormalTok{] }\SpecialCharTok{+}\NormalTok{ state.x77[,}\StringTok{"Illiteracy"}\NormalTok{] }\SpecialCharTok{+}\NormalTok{ state.x77[,}\StringTok{"Life Exp"}\NormalTok{])}

\FunctionTok{plot}\NormalTok{(state.x77[,}\StringTok{"Murder"}\NormalTok{], }\FunctionTok{predict}\NormalTok{(model), }\AttributeTok{main =} \StringTok{"Linear Regression"}\NormalTok{, }\AttributeTok{xlab =} \StringTok{"Tatsaechliche Werte"}\NormalTok{, }\AttributeTok{ylab =} \StringTok{"Vorhergesehene Werte"}\NormalTok{)}
\FunctionTok{abline}\NormalTok{(}\DecValTok{0}\NormalTok{,}\DecValTok{1}\NormalTok{, }\AttributeTok{col =} \StringTok{"red"}\NormalTok{)}
\end{Highlighting}
\end{Shaded}

\includegraphics{StatistischeModellierung_files/figure-latex/unnamed-chunk-1-1.pdf}

\begin{Shaded}
\begin{Highlighting}[]
\NormalTok{coef }\OtherTok{\textless{}{-}} \FunctionTok{coef}\NormalTok{(model)}
\NormalTok{model\_equation }\OtherTok{\textless{}{-}} \FunctionTok{paste}\NormalTok{(}\StringTok{"Y="}\NormalTok{,}\FunctionTok{round}\NormalTok{(coef[}\DecValTok{1}\NormalTok{],}\DecValTok{2}\NormalTok{), }\StringTok{"+"}\NormalTok{,}
                            \FunctionTok{round}\NormalTok{(coef[}\DecValTok{2}\NormalTok{],}\DecValTok{4}\NormalTok{),}\StringTok{"* X1 +"}\NormalTok{,}
                            \FunctionTok{round}\NormalTok{(coef[}\DecValTok{3}\NormalTok{],}\DecValTok{4}\NormalTok{),}\StringTok{"* X2 +"}\NormalTok{,}
                            \FunctionTok{round}\NormalTok{(coef[}\DecValTok{4}\NormalTok{],}\DecValTok{2}\NormalTok{),}\StringTok{"* X3 +"}\NormalTok{,}
                            \FunctionTok{round}\NormalTok{(coef[}\DecValTok{5}\NormalTok{],}\DecValTok{2}\NormalTok{),}\StringTok{"* X4"}\NormalTok{)}

\FunctionTok{print}\NormalTok{(model\_equation)}
\end{Highlighting}
\end{Shaded}

\begin{verbatim}
## [1] "Y= 112.84 + 2e-04 * X1 + 5e-04 * X2 + 2.27 * X3 + -1.57 * X4"
\end{verbatim}

Aus der resultierenden Regressionsformel laest sich erkennen, dass die
Illiteracy den groessten Einfluss auf die Mordrate hat. Den kleinsten
Einfluss hat die Population. Eine niedrige Murder rate bringt eine
hoehere Life Exp.

\subsubsection{Führe alle fünf für dieses Regressionsmodell geltenden
Modellvoraussetzungen an und überprüfe diese Voraussetzungen
nachweislich anhand der Zusammenfassung (summary), Quality Plots der
Regression und der pairwise Scatterplot Matrix. Erkläre, ob diese Modell
überhaupt gültig ist. Falls es gültig ist, gib die Qualität der
Erklärung durch das Modell
an.}\label{fuxfchre-alle-fuxfcnf-fuxfcr-dieses-regressionsmodell-geltenden-modellvoraussetzungen-an-und-uxfcberpruxfcfe-diese-voraussetzungen-nachweislich-anhand-der-zusammenfassung-summary-quality-plots-der-regression-und-der-pairwise-scatterplot-matrix.-erkluxe4re-ob-diese-modell-uxfcberhaupt-guxfcltig-ist.-falls-es-guxfcltig-ist-gib-die-qualituxe4t-der-erkluxe4rung-durch-das-modell-an.}

\begin{Shaded}
\begin{Highlighting}[]
\NormalTok{model }\OtherTok{\textless{}{-}}  \FunctionTok{lm}\NormalTok{(state.x77[,}\StringTok{"Murder"}\NormalTok{] }\SpecialCharTok{\textasciitilde{}}\NormalTok{ state.x77[,}\StringTok{"Population"}\NormalTok{] }\SpecialCharTok{+}\NormalTok{ state.x77[,}\StringTok{"Income"}\NormalTok{] }\SpecialCharTok{+}\NormalTok{ state.x77[,}\StringTok{"Illiteracy"}\NormalTok{] }\SpecialCharTok{+}\NormalTok{ state.x77[,}\StringTok{"Life Exp"}\NormalTok{])}
\CommentTok{\# Korrelation}
\FunctionTok{summary}\NormalTok{(model)}
\end{Highlighting}
\end{Shaded}

\begin{verbatim}
## 
## Call:
## lm(formula = state.x77[, "Murder"] ~ state.x77[, "Population"] + 
##     state.x77[, "Income"] + state.x77[, "Illiteracy"] + state.x77[, 
##     "Life Exp"])
## 
## Residuals:
##    Min     1Q Median     3Q    Max 
## -3.387 -1.116  0.105  1.478  3.249 
## 
## Coefficients:
##                             Estimate Std. Error t value Pr(>|t|)    
## (Intercept)                1.128e+02  1.740e+01   6.487 5.90e-08 ***
## state.x77[, "Population"]  2.059e-04  6.131e-05   3.358 0.001606 ** 
## state.x77[, "Income"]      4.524e-04  4.956e-04   0.913 0.366230    
## state.x77[, "Illiteracy"]  2.265e+00  5.651e-01   4.008 0.000227 ***
## state.x77[, "Life Exp"]   -1.566e+00  2.419e-01  -6.474 6.15e-08 ***
## ---
## Signif. codes:  0 '***' 0.001 '**' 0.01 '*' 0.05 '.' 0.1 ' ' 1
## 
## Residual standard error: 1.824 on 45 degrees of freedom
## Multiple R-squared:  0.7758, Adjusted R-squared:  0.7558 
## F-statistic: 38.92 on 4 and 45 DF,  p-value: 4.532e-14
\end{verbatim}

\begin{Shaded}
\begin{Highlighting}[]
\CommentTok{\# Systematische Fehler}
\FunctionTok{plot}\NormalTok{(model, }\AttributeTok{which =} \DecValTok{1}\NormalTok{)}
\end{Highlighting}
\end{Shaded}

\includegraphics{StatistischeModellierung_files/figure-latex/unnamed-chunk-2-1.pdf}

\begin{Shaded}
\begin{Highlighting}[]
\CommentTok{\# homoskedastitzitaet}
\FunctionTok{plot}\NormalTok{(model, }\AttributeTok{which =} \DecValTok{3}\NormalTok{)}
\end{Highlighting}
\end{Shaded}

\includegraphics{StatistischeModellierung_files/figure-latex/unnamed-chunk-2-2.pdf}

\begin{Shaded}
\begin{Highlighting}[]
\CommentTok{\# Modellfehler normalverteilt}
\FunctionTok{plot}\NormalTok{(model, }\AttributeTok{which =} \DecValTok{2}\NormalTok{)}
\end{Highlighting}
\end{Shaded}

\includegraphics{StatistischeModellierung_files/figure-latex/unnamed-chunk-2-3.pdf}

\begin{Shaded}
\begin{Highlighting}[]
\CommentTok{\# multikollinearitaet}
\NormalTok{p }\OtherTok{\textless{}{-}}\NormalTok{ state.x77[,}\FunctionTok{c}\NormalTok{(}\StringTok{"Population"}\NormalTok{,}\StringTok{"Income"}\NormalTok{, }\StringTok{"Illiteracy"}\NormalTok{, }\StringTok{"Life Exp"}\NormalTok{, }\StringTok{"Murder"}\NormalTok{)]}
\FunctionTok{print}\NormalTok{(}\FunctionTok{cor}\NormalTok{(p))}
\end{Highlighting}
\end{Shaded}

\begin{verbatim}
##             Population     Income Illiteracy    Life Exp     Murder
## Population  1.00000000  0.2082276  0.1076224 -0.06805195  0.3436428
## Income      0.20822756  1.0000000 -0.4370752  0.34025534 -0.2300776
## Illiteracy  0.10762237 -0.4370752  1.0000000 -0.58847793  0.7029752
## Life Exp   -0.06805195  0.3402553 -0.5884779  1.00000000 -0.7808458
## Murder      0.34364275 -0.2300776  0.7029752 -0.78084575  1.0000000
\end{verbatim}

\begin{Shaded}
\begin{Highlighting}[]
\NormalTok{panel.hist }\OtherTok{\textless{}{-}} \ControlFlowTok{function}\NormalTok{(x, ...) \{}
\NormalTok{    usr }\OtherTok{\textless{}{-}} \FunctionTok{par}\NormalTok{(}\StringTok{"usr"}\NormalTok{); }\FunctionTok{on.exit}\NormalTok{(}\FunctionTok{par}\NormalTok{(usr))}
    \FunctionTok{par}\NormalTok{(}\AttributeTok{usr =} \FunctionTok{c}\NormalTok{(usr[}\DecValTok{1}\SpecialCharTok{:}\DecValTok{2}\NormalTok{], }\DecValTok{0}\NormalTok{, }\FloatTok{1.5}\NormalTok{))}
\NormalTok{    h }\OtherTok{\textless{}{-}} \FunctionTok{hist}\NormalTok{(x, }\AttributeTok{plot =} \ConstantTok{FALSE}\NormalTok{)}
\NormalTok{    breaks }\OtherTok{\textless{}{-}}\NormalTok{ h}\SpecialCharTok{$}\NormalTok{breaks; nB }\OtherTok{\textless{}{-}} \FunctionTok{length}\NormalTok{(breaks)}
\NormalTok{    y }\OtherTok{\textless{}{-}}\NormalTok{ h}\SpecialCharTok{$}\NormalTok{counts; y }\OtherTok{\textless{}{-}}\NormalTok{ y}\SpecialCharTok{/}\FunctionTok{max}\NormalTok{(y)}
    \FunctionTok{rect}\NormalTok{(breaks[}\SpecialCharTok{{-}}\NormalTok{nB], }\DecValTok{0}\NormalTok{, breaks[}\SpecialCharTok{{-}}\DecValTok{1}\NormalTok{], y, }\AttributeTok{col =} \StringTok{"cyan"}\NormalTok{, ...)}
\NormalTok{\}}

\NormalTok{panel.cor }\OtherTok{\textless{}{-}} \ControlFlowTok{function}\NormalTok{(x, y, }\AttributeTok{digits =} \DecValTok{2}\NormalTok{, }\AttributeTok{prefix =} \StringTok{""}\NormalTok{, cex.cor, ...) \{}
\NormalTok{    usr }\OtherTok{\textless{}{-}} \FunctionTok{par}\NormalTok{(}\StringTok{"usr"}\NormalTok{); }\FunctionTok{on.exit}\NormalTok{(}\FunctionTok{par}\NormalTok{(usr))}
    \FunctionTok{par}\NormalTok{(}\AttributeTok{usr =} \FunctionTok{c}\NormalTok{(}\DecValTok{0}\NormalTok{, }\DecValTok{1}\NormalTok{, }\DecValTok{0}\NormalTok{, }\DecValTok{1}\NormalTok{))}
\NormalTok{    r }\OtherTok{\textless{}{-}} \FunctionTok{abs}\NormalTok{(}\FunctionTok{cor}\NormalTok{(x, y))}
\NormalTok{    txt }\OtherTok{\textless{}{-}} \FunctionTok{format}\NormalTok{(}\FunctionTok{c}\NormalTok{(r, }\FloatTok{0.123456789}\NormalTok{), }\AttributeTok{digits =}\NormalTok{ digits)[}\DecValTok{1}\NormalTok{]}
\NormalTok{    txt }\OtherTok{\textless{}{-}} \FunctionTok{paste0}\NormalTok{(prefix, txt)}
    \ControlFlowTok{if}\NormalTok{(}\FunctionTok{missing}\NormalTok{(cex.cor)) cex.cor }\OtherTok{\textless{}{-}} \FloatTok{0.8}\SpecialCharTok{/}\FunctionTok{strwidth}\NormalTok{(txt)}
    \FunctionTok{text}\NormalTok{(}\FloatTok{0.5}\NormalTok{, }\FloatTok{0.5}\NormalTok{, txt, }\AttributeTok{cex =}\NormalTok{ cex.cor }\SpecialCharTok{*}\NormalTok{ r)}
\NormalTok{\}}

\CommentTok{\# Create a matrix of scatterplots}
\FunctionTok{pairs}\NormalTok{(state.x77[, }\FunctionTok{c}\NormalTok{(}\StringTok{"Life Exp"}\NormalTok{ , }\StringTok{"Population"}\NormalTok{ , }\StringTok{"Illiteracy"}\NormalTok{, }\StringTok{"Income"}\NormalTok{, }\StringTok{"Murder"}\NormalTok{)], }
      \AttributeTok{lower.panel =}\NormalTok{ panel.smooth, }
      \AttributeTok{upper.panel =}\NormalTok{ panel.cor, }
      \AttributeTok{diag.panel =}\NormalTok{ panel.hist, }
      \AttributeTok{las=}\DecValTok{1}\NormalTok{)}
\end{Highlighting}
\end{Shaded}

\begin{verbatim}
## Warning in par(usr): argument 1 does not name a graphical parameter

## Warning in par(usr): argument 1 does not name a graphical parameter

## Warning in par(usr): argument 1 does not name a graphical parameter

## Warning in par(usr): argument 1 does not name a graphical parameter

## Warning in par(usr): argument 1 does not name a graphical parameter

## Warning in par(usr): argument 1 does not name a graphical parameter

## Warning in par(usr): argument 1 does not name a graphical parameter

## Warning in par(usr): argument 1 does not name a graphical parameter

## Warning in par(usr): argument 1 does not name a graphical parameter

## Warning in par(usr): argument 1 does not name a graphical parameter

## Warning in par(usr): argument 1 does not name a graphical parameter

## Warning in par(usr): argument 1 does not name a graphical parameter

## Warning in par(usr): argument 1 does not name a graphical parameter

## Warning in par(usr): argument 1 does not name a graphical parameter

## Warning in par(usr): argument 1 does not name a graphical parameter
\end{verbatim}

\includegraphics{StatistischeModellierung_files/figure-latex/unnamed-chunk-2-4.pdf}

Es liegen keine systematischen Fehler vor da die Fehlervarianz konstant
ist, es eine kollinearitaet zwischen der Illiteracy und der Life Exp,
Murder rate, Income, alle Werte im QQ-Pot liegen nahe an der Geraden
somit sind die Modellfehler Nomalverteilt. -\textgreater{} Modell
sinvoll

\subsubsection{Führe eine Modellselektion der relevanten erklärenden
Variablen
durch.}\label{fuxfchre-eine-modellselektion-der-relevanten-erkluxe4renden-variablen-durch.}

\begin{Shaded}
\begin{Highlighting}[]
\NormalTok{model }\OtherTok{\textless{}{-}}  \FunctionTok{lm}\NormalTok{(state.x77[,}\StringTok{"Murder"}\NormalTok{] }\SpecialCharTok{\textasciitilde{}}\NormalTok{ state.x77[,}\StringTok{"Population"}\NormalTok{] }\SpecialCharTok{+}\NormalTok{ state.x77[,}\StringTok{"Income"}\NormalTok{] }\SpecialCharTok{+}\NormalTok{ state.x77[,}\StringTok{"Illiteracy"}\NormalTok{] }\SpecialCharTok{+}\NormalTok{ state.x77[,}\StringTok{"Life Exp"}\NormalTok{])}
\CommentTok{\# Korrelation}
\FunctionTok{summary}\NormalTok{(model)}
\end{Highlighting}
\end{Shaded}

\begin{verbatim}
## 
## Call:
## lm(formula = state.x77[, "Murder"] ~ state.x77[, "Population"] + 
##     state.x77[, "Income"] + state.x77[, "Illiteracy"] + state.x77[, 
##     "Life Exp"])
## 
## Residuals:
##    Min     1Q Median     3Q    Max 
## -3.387 -1.116  0.105  1.478  3.249 
## 
## Coefficients:
##                             Estimate Std. Error t value Pr(>|t|)    
## (Intercept)                1.128e+02  1.740e+01   6.487 5.90e-08 ***
## state.x77[, "Population"]  2.059e-04  6.131e-05   3.358 0.001606 ** 
## state.x77[, "Income"]      4.524e-04  4.956e-04   0.913 0.366230    
## state.x77[, "Illiteracy"]  2.265e+00  5.651e-01   4.008 0.000227 ***
## state.x77[, "Life Exp"]   -1.566e+00  2.419e-01  -6.474 6.15e-08 ***
## ---
## Signif. codes:  0 '***' 0.001 '**' 0.01 '*' 0.05 '.' 0.1 ' ' 1
## 
## Residual standard error: 1.824 on 45 degrees of freedom
## Multiple R-squared:  0.7758, Adjusted R-squared:  0.7558 
## F-statistic: 38.92 on 4 and 45 DF,  p-value: 4.532e-14
\end{verbatim}

\begin{Shaded}
\begin{Highlighting}[]
\CommentTok{\# Systematische Fehler}
\FunctionTok{plot}\NormalTok{(model, }\AttributeTok{which =} \DecValTok{1}\NormalTok{)}
\end{Highlighting}
\end{Shaded}

\includegraphics{StatistischeModellierung_files/figure-latex/unnamed-chunk-3-1.pdf}

\begin{Shaded}
\begin{Highlighting}[]
\CommentTok{\# homoskedastitzitaet}
\FunctionTok{plot}\NormalTok{(model, }\AttributeTok{which =} \DecValTok{3}\NormalTok{)}
\end{Highlighting}
\end{Shaded}

\includegraphics{StatistischeModellierung_files/figure-latex/unnamed-chunk-3-2.pdf}

\begin{Shaded}
\begin{Highlighting}[]
\CommentTok{\# Modellfehler normalverteilt}
\FunctionTok{plot}\NormalTok{(model, }\AttributeTok{which =} \DecValTok{2}\NormalTok{)}
\end{Highlighting}
\end{Shaded}

\includegraphics{StatistischeModellierung_files/figure-latex/unnamed-chunk-3-3.pdf}

\begin{Shaded}
\begin{Highlighting}[]
\CommentTok{\# multikollinearitaet}
\NormalTok{p }\OtherTok{\textless{}{-}}\NormalTok{ state.x77[,}\FunctionTok{c}\NormalTok{(}\StringTok{"Population"}\NormalTok{, }\StringTok{"Illiteracy"}\NormalTok{)]}
\FunctionTok{print}\NormalTok{(}\FunctionTok{cor}\NormalTok{(p))}
\end{Highlighting}
\end{Shaded}

\begin{verbatim}
##            Population Illiteracy
## Population  1.0000000  0.1076224
## Illiteracy  0.1076224  1.0000000
\end{verbatim}

\begin{Shaded}
\begin{Highlighting}[]
\NormalTok{panel.hist }\OtherTok{\textless{}{-}} \ControlFlowTok{function}\NormalTok{(x, ...) \{}
\NormalTok{    usr }\OtherTok{\textless{}{-}} \FunctionTok{par}\NormalTok{(}\StringTok{"usr"}\NormalTok{); }\FunctionTok{on.exit}\NormalTok{(}\FunctionTok{par}\NormalTok{(usr))}
    \FunctionTok{par}\NormalTok{(}\AttributeTok{usr =} \FunctionTok{c}\NormalTok{(usr[}\DecValTok{1}\SpecialCharTok{:}\DecValTok{2}\NormalTok{], }\DecValTok{0}\NormalTok{, }\FloatTok{1.5}\NormalTok{))}
\NormalTok{    h }\OtherTok{\textless{}{-}} \FunctionTok{hist}\NormalTok{(x, }\AttributeTok{plot =} \ConstantTok{FALSE}\NormalTok{)}
\NormalTok{    breaks }\OtherTok{\textless{}{-}}\NormalTok{ h}\SpecialCharTok{$}\NormalTok{breaks; nB }\OtherTok{\textless{}{-}} \FunctionTok{length}\NormalTok{(breaks)}
\NormalTok{    y }\OtherTok{\textless{}{-}}\NormalTok{ h}\SpecialCharTok{$}\NormalTok{counts; y }\OtherTok{\textless{}{-}}\NormalTok{ y}\SpecialCharTok{/}\FunctionTok{max}\NormalTok{(y)}
    \FunctionTok{rect}\NormalTok{(breaks[}\SpecialCharTok{{-}}\NormalTok{nB], }\DecValTok{0}\NormalTok{, breaks[}\SpecialCharTok{{-}}\DecValTok{1}\NormalTok{], y, }\AttributeTok{col =} \StringTok{"cyan"}\NormalTok{, ...)}
\NormalTok{\}}

\NormalTok{panel.cor }\OtherTok{\textless{}{-}} \ControlFlowTok{function}\NormalTok{(x, y, }\AttributeTok{digits =} \DecValTok{2}\NormalTok{, }\AttributeTok{prefix =} \StringTok{""}\NormalTok{, cex.cor, ...) \{}
\NormalTok{    usr }\OtherTok{\textless{}{-}} \FunctionTok{par}\NormalTok{(}\StringTok{"usr"}\NormalTok{); }\FunctionTok{on.exit}\NormalTok{(}\FunctionTok{par}\NormalTok{(usr))}
    \FunctionTok{par}\NormalTok{(}\AttributeTok{usr =} \FunctionTok{c}\NormalTok{(}\DecValTok{0}\NormalTok{, }\DecValTok{1}\NormalTok{, }\DecValTok{0}\NormalTok{, }\DecValTok{1}\NormalTok{))}
\NormalTok{    r }\OtherTok{\textless{}{-}} \FunctionTok{abs}\NormalTok{(}\FunctionTok{cor}\NormalTok{(x, y))}
\NormalTok{    txt }\OtherTok{\textless{}{-}} \FunctionTok{format}\NormalTok{(}\FunctionTok{c}\NormalTok{(r, }\FloatTok{0.123456789}\NormalTok{), }\AttributeTok{digits =}\NormalTok{ digits)[}\DecValTok{1}\NormalTok{]}
\NormalTok{    txt }\OtherTok{\textless{}{-}} \FunctionTok{paste0}\NormalTok{(prefix, txt)}
    \ControlFlowTok{if}\NormalTok{(}\FunctionTok{missing}\NormalTok{(cex.cor)) cex.cor }\OtherTok{\textless{}{-}} \FloatTok{0.8}\SpecialCharTok{/}\FunctionTok{strwidth}\NormalTok{(txt)}
    \FunctionTok{text}\NormalTok{(}\FloatTok{0.5}\NormalTok{, }\FloatTok{0.5}\NormalTok{, txt, }\AttributeTok{cex =}\NormalTok{ cex.cor }\SpecialCharTok{*}\NormalTok{ r)}
\NormalTok{\}}

\CommentTok{\# Create a matrix of scatterplots}
\FunctionTok{pairs}\NormalTok{(state.x77[, }\FunctionTok{c}\NormalTok{(}\StringTok{"Population"}\NormalTok{, }\StringTok{"Illiteracy"}\NormalTok{)], }
      \AttributeTok{lower.panel =}\NormalTok{ panel.smooth, }
      \AttributeTok{upper.panel =}\NormalTok{ panel.cor, }
      \AttributeTok{diag.panel =}\NormalTok{ panel.hist, }
      \AttributeTok{las=}\DecValTok{1}\NormalTok{)}
\end{Highlighting}
\end{Shaded}

\begin{verbatim}
## Warning in par(usr): argument 1 does not name a graphical parameter

## Warning in par(usr): argument 1 does not name a graphical parameter

## Warning in par(usr): argument 1 does not name a graphical parameter
\end{verbatim}

\includegraphics{StatistischeModellierung_files/figure-latex/unnamed-chunk-3-4.pdf}

\subsection{Aufgabe 2}\label{aufgabe-2}

\subsubsection{Installiere das Package `MASS' mithilfe der Funktion
install.packages. Lade den Datensatz `Pima.tr' in R. Beschreibe die
Daten anhand der internen
Hilfe.}\label{installiere-das-package-mass-mithilfe-der-funktion-install.packages.-lade-den-datensatz-pima.tr-in-r.-beschreibe-die-daten-anhand-der-internen-hilfe.}

\begin{Shaded}
\begin{Highlighting}[]
\FunctionTok{library}\NormalTok{(MASS)}
\FunctionTok{data}\NormalTok{(Pima.tr)}
\NormalTok{?Pima.tr}
\end{Highlighting}
\end{Shaded}

Pima.tr ist ein Datensatz welcher Daten ueber Indische Frauen ueber 21
welche in der naehe von Phoenix Arizona wohnen beinhaltet. Die
beinhalteten Daten sind:

\begin{itemize}
\tightlist
\item
  npreg
\end{itemize}

\begin{verbatim}
-    Die anzahl an Schwangerschaften
\end{verbatim}

\begin{itemize}
\item
  glu

  \begin{itemize}
  \tightlist
  \item
    Plasma glucose konzentration
  \end{itemize}
\item
  bp

  \begin{itemize}
  \tightlist
  \item
    Blutdruck
  \end{itemize}
\item
  skin

  \begin{itemize}
  \tightlist
  \item
    Die dicke der Haut am triceps
  \end{itemize}
\item
  bmi

  \begin{itemize}
  \tightlist
  \item
    Body mass index
  \end{itemize}
\item
  ped

  \begin{itemize}
  \tightlist
  \item
    Diabetes-Stammbaumfunktion
  \end{itemize}
\item
  age

  \begin{itemize}
  \tightlist
  \item
    Alter
  \end{itemize}
\item
  type

  \begin{itemize}
  \tightlist
  \item
    Ob die Person von der WHO gesehen diabetis hat.
  \end{itemize}
\end{itemize}

\subsection{Ermittle ein logistisches Regressionsmodell, dass das
Auftreten von Diabetes (`type') durch die übrigen unabhängigen Variablen
Alter (age), Anzahl der Schwangerschaften (npreg), BMI, Glukosespiegel
(glu), Blutdruck (bp), familiäre Häufung von Diabetesfällen (ped) und
Hautfaltendickemessung am Oberarm (skin) erklärt. Schreibe die
Modellgleichung an und interpretiere die Werte der Koeffizienten im
Kontext.}\label{ermittle-ein-logistisches-regressionsmodell-dass-das-auftreten-von-diabetes-type-durch-die-uxfcbrigen-unabhuxe4ngigen-variablen-alter-age-anzahl-der-schwangerschaften-npreg-bmi-glukosespiegel-glu-blutdruck-bp-familiuxe4re-huxe4ufung-von-diabetesfuxe4llen-ped-und-hautfaltendickemessung-am-oberarm-skin-erkluxe4rt.-schreibe-die-modellgleichung-an-und-interpretiere-die-werte-der-koeffizienten-im-kontext.}

\begin{Shaded}
\begin{Highlighting}[]
\FunctionTok{library}\NormalTok{(MASS)}
\FunctionTok{library}\NormalTok{(ggplot2)}

\FunctionTok{data}\NormalTok{(Pima.tr)}

\NormalTok{model }\OtherTok{\textless{}{-}} \FunctionTok{glm}\NormalTok{(Pima.tr}\SpecialCharTok{$}\NormalTok{type }\SpecialCharTok{\textasciitilde{}}\NormalTok{ Pima.tr}\SpecialCharTok{$}\NormalTok{npreg }\SpecialCharTok{+}\NormalTok{ Pima.tr}\SpecialCharTok{$}\NormalTok{glu }\SpecialCharTok{+}\NormalTok{ Pima.tr}\SpecialCharTok{$}\NormalTok{bp }\SpecialCharTok{+}\NormalTok{ Pima.tr}\SpecialCharTok{$}\NormalTok{skin }\SpecialCharTok{+}\NormalTok{ Pima.tr}\SpecialCharTok{$}\NormalTok{bmi }\SpecialCharTok{+}\NormalTok{ Pima.tr}\SpecialCharTok{$}\NormalTok{ped }\SpecialCharTok{+}\NormalTok{ Pima.tr}\SpecialCharTok{$}\NormalTok{age, }\AttributeTok{data =}\NormalTok{ Pima.tr, }\AttributeTok{family =}\NormalTok{ binomial)}

\NormalTok{coef }\OtherTok{\textless{}{-}} \FunctionTok{coef}\NormalTok{(model)}
\NormalTok{model\_equation }\OtherTok{\textless{}{-}} \FunctionTok{paste}\NormalTok{(}\StringTok{"Y = "}\NormalTok{,}\FunctionTok{round}\NormalTok{(coef[}\DecValTok{1}\NormalTok{],}\DecValTok{2}\NormalTok{), }\StringTok{" + "}\NormalTok{,}
                            \FunctionTok{round}\NormalTok{(coef[}\DecValTok{3}\NormalTok{],}\DecValTok{2}\NormalTok{),}\StringTok{" * glu + "}\NormalTok{,}
                            \FunctionTok{round}\NormalTok{(coef[}\DecValTok{4}\NormalTok{],}\DecValTok{2}\NormalTok{),}\StringTok{" * bmi + "}\NormalTok{,}
                            \FunctionTok{round}\NormalTok{(coef[}\DecValTok{5}\NormalTok{],}\DecValTok{2}\NormalTok{),}\StringTok{" * ped + "}\NormalTok{,}
                            \FunctionTok{round}\NormalTok{(coef[}\DecValTok{6}\NormalTok{],}\DecValTok{4}\NormalTok{),}\StringTok{" * bp + "}\NormalTok{,}
                            \FunctionTok{round}\NormalTok{(coef[}\DecValTok{7}\NormalTok{],}\DecValTok{2}\NormalTok{),}\StringTok{" * age"}\NormalTok{)}

\FunctionTok{summary}\NormalTok{(model)}
\end{Highlighting}
\end{Shaded}

\begin{verbatim}
## 
## Call:
## glm(formula = Pima.tr$type ~ Pima.tr$npreg + Pima.tr$glu + Pima.tr$bp + 
##     Pima.tr$skin + Pima.tr$bmi + Pima.tr$ped + Pima.tr$age, family = binomial, 
##     data = Pima.tr)
## 
## Coefficients:
##                Estimate Std. Error z value Pr(>|z|)    
## (Intercept)   -9.773062   1.770386  -5.520 3.38e-08 ***
## Pima.tr$npreg  0.103183   0.064694   1.595  0.11073    
## Pima.tr$glu    0.032117   0.006787   4.732 2.22e-06 ***
## Pima.tr$bp    -0.004768   0.018541  -0.257  0.79707    
## Pima.tr$skin  -0.001917   0.022500  -0.085  0.93211    
## Pima.tr$bmi    0.083624   0.042827   1.953  0.05087 .  
## Pima.tr$ped    1.820410   0.665514   2.735  0.00623 ** 
## Pima.tr$age    0.041184   0.022091   1.864  0.06228 .  
## ---
## Signif. codes:  0 '***' 0.001 '**' 0.01 '*' 0.05 '.' 0.1 ' ' 1
## 
## (Dispersion parameter for binomial family taken to be 1)
## 
##     Null deviance: 256.41  on 199  degrees of freedom
## Residual deviance: 178.39  on 192  degrees of freedom
## AIC: 194.39
## 
## Number of Fisher Scoring iterations: 5
\end{verbatim}

\begin{Shaded}
\begin{Highlighting}[]
\FunctionTok{print}\NormalTok{(model\_equation)}
\end{Highlighting}
\end{Shaded}

\begin{verbatim}
## [1] "Y =  -9.77  +  0.03  * glu +  0  * bmi +  0  * ped +  0.0836  * bp +  1.82  * age"
\end{verbatim}

\subsubsection{Begriffe}\label{begriffe}

\begin{itemize}
\tightlist
\item
  Scatterplot Matrix
\item
  Zeigt mehrere Streudiagramme. Bei mehreren variabeln kann die
  korrelation darstellen.
\item
  lineare Regression
\item
  Stellt eine Gleichung auf welche moeglichst genau durch alle
  Datenpunkte geht.
\item
  Quality Plots
\item
  Hilft die Qualitaet eines Modells zu ueberpruefen.
\item
  Residuen
\item
  Die Differenz zwischen vorhergesagten Werten und tatsaechlichen
  Werten.
\item
  Regressionskoeffizienten
\item
  Parameter einer Regressionsgleichung
\item
  Regressionsmodell
\item
  Zusammenhand zwischen anhaengiger und unabhaengigen variabeln.
\item
  Modellgleichung
\item
  Gleich wie Regressionsmodell
\item
  logistische Regression
\item
  Modell fuer binaere abhaengige variabeln.
\end{itemize}

\end{document}
